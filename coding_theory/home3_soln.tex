\documentclass[12pt]{article}

\usepackage[utf8]{inputenc}
\usepackage{latexsym,amsfonts,amssymb,amsthm,amsmath}

\setlength{\parindent}{0in}
\setlength{\oddsidemargin}{0in}
\setlength{\textwidth}{6.5in}
\setlength{\textheight}{8.8in}
\setlength{\topmargin}{0in}
\setlength{\headheight}{18pt}



\title{Coding Theory - Homework 3}
\author{Kishlaya Jaiswal}

\begin{document}

\maketitle

\vspace{0.5in}



\subsection*{Exercise 1}

\begin{proof}
Since $f$ is a boolean function, $f: \mathbb{F}^m \to \mathbb{F}$. Fix $u \in \mathbb{F}^m$, then we have two cases:

\begin{itemize}
    \item $f(u) = 1$, in which case, we replace $1$ by $-1$ and note that $(-1)^{f(u)} = (-1)^1 = -1 = F(u)$.
    \item $f(u) = 0$, in which case, we replace $0$ by $1$ and note that $(-1)^{f(u)} = (-1)^0 = 1 = F(u)$.
\end{itemize}
Hence, $F(u) = (-1)^{f(u)}$.
\end{proof}

\subsection*{Exercise 2}
\begin{proof}
$F$ is a row vector whose $u^{\text{th}}$ co-ordinate is $F(u)$. Hence $$(FH)_u = \sum_{v \in \mathbb{F}^m} F_v H_{v,u} = \sum_{v \in \mathbb{F}^m} F(v)(-1)^{v.u} = \sum_{v \in \mathbb{F}^m} (-1)^{u.v + f(v)}$$
Hence, $\tilde{F} = FH$.
\end{proof}

\subsection*{Exercise 3}
\begin{proof}
We have $H_{u,v} = (-1)^{u.v}$ which is a symmetric Hadamard symmetric matrix of order $2^m$. We get,
\begin{itemize}
    \item $H^t = H$, as $H$ is symmetric
    \item $H^t H = 2^m I_{2^m}$ as $H$ is Hadamard matrix.
\end{itemize}
Hence, $H^{-1} = \frac{1}{2^m} H^t = \frac{1}{2^m} H$. Therefore,
$$\tilde{F} = FH \implies F = \frac{1}{2^m}\tilde{F}H \implies F(v) = \frac{1}{2^m} \sum_u \tilde{F}(u) H_{u,v} = \frac{1}{2^m} \sum_u (-1)^{u.v}\tilde{F}(u)$$
\end{proof}

\subsection*{Exercise 4}
\begin{proof}
Let $X(u)$ be a row vector whose $v^{\text{th}}$ co-ordinate is $f(v) + u.v$.

We observe that, 
$$\tilde{F}(u) = \sum_{v \in \mathbb{F}^m} (-1)^{u.v+f(v)} = \sum_{v \in \mathbb{F}^m} (-1)^{X(u)_v} = \sum_{v, X(u)_v = 0} (1) + \sum_{v, X(u)_v = 1} (-1)$$
Hence $\tilde{F}(u) = $ difference between \# of zeroes and \# of ones in $X(u)$.
\end{proof}

\subsection*{Exercise 5}
\begin{proof}
Denote by 
\begin{itemize}
    \item $\alpha_{00}$ the number of $v$ for which $f(v) = u.v = 0$
    \item $\alpha_{01}$ the number of $v$ for which $f(v) = 0$ and $u.v = 1$
    \item $\alpha_{10}$ the number of $v$ for which $f(v) = 1$ and $u.v = 0$
    \item $\alpha_{11}$ the number of $v$ for which $f(v) = u.v = 1$
\end{itemize}
Since this list is exhaustive, we know that $\alpha_{00} + \alpha_{01} + \alpha_{10} + \alpha_{11} = \text{length}(f) = 2^m$.

Next note that, $d(f, u.v) = f - u.v = $ weight of the vector $(f + u.v) = \alpha_{01} + \alpha_{10}$.

As computed above, $\tilde{F}(u)$ is the difference between \# of zeroes and \# of ones in $X(u) = f + u.v$, so $$\tilde{F}(u) = \alpha_{00} + \alpha_{11} - \alpha_{01} - \alpha_{10} = \alpha_{00} + \alpha_{11} + \alpha_{01} + \alpha_{10} - 2(\alpha_{01} + \alpha_{10}) = 2^m - 2d(f,u.v)$$ as desired.
\end{proof}

\subsection*{Exercise 6}
\begin{proof}
We first note that $$\sum_v (-1)^{u.v + f(v) + 1} = - \sum_v (-1)^{u.v + f(v)} = -\tilde{F}(u)$$
Hence, the Walsh transform of $\bar{f} = 1+f$ is $- \tilde{F}$.

Now let $C$ be the coset of $R(1,m)$ containing $f$. 

Let $x \in C$. Then $x-f \in R(1,m)$. Since $\{\hat{1}, v_0, \ldots v_{m-1}\}$ is a basis for $R(1,m)$, $x = f + c\hat{1} + c_ov_0 + \cdots + c_{m-1}v_{m-1}$. 

Here we make a clever observation. When $c_ov_0 + \cdots + c_{m-1}v_{m-1}$ is written in matrix notation as: 
$$\begin{pmatrix}
c_0 & c_1 & c_2 & \ldots & c_{m-1} \\
\end{pmatrix} \begin{pmatrix}
0 & 1 & 0 & 1 & \ldots & 0 & 1 & 0 & 1 \\
0 & 0 & 1 & 1 & \ldots & 0 & 0 & 1 & 1 \\
\vdots & \vdots & \vdots & \vdots & \ddots & \vdots & \vdots & \vdots & \vdots \\
0 & 0 & 0 & 0 & \ldots & 1 & 1 & 1 & 1 \\
\end{pmatrix}$$

can be viewed as $(c.v)_{v \in \mathbb{F}^m}$. Hence $c_ov_0 + \cdots + c_{m-1}v_{m-1} = (c.v)_v$.

If $c=0$ (coefficient of $\hat{1}$ is $0$), $x = f + c.v$ and so $wt(x) = d(x, 0) = d(f, c.v) = \frac{1}{2}(2^m - \tilde{F}(c))$

otherwise $c=1$, $x = \bar{f} + c.v$ and so $wt(x) = d(x, 0) = d(\bar{f}, c.v) = \frac{1}{2}(2^m + \tilde{F}(c))$.

Hence, weight of each word contained in the coset $C$ is $$\frac{1}{2}(2^m \pm \tilde{F}(u))$$
for $u \in \mathbb{F}^m$.
\end{proof}

\subsection*{Exercise 7}
\begin{proof}
Let us first consider the number $x(a,b) = 1 + (-1)^a + (-1)^b + (-1)^{a+b+1}$. We shall show that $x(a,b) = \pm 2$ $\forall (a,b) \in \mathbb{F}_2^2$.
\begin{itemize}
    \item $x(0,0) = 1+1+1-1 = 2$
    \item $x(0,1) = 1+1-1+1 = 2$
    \item $x(1,0) = 1-1+1+1 = 2$
    \item $x(1,1) = 1-1-1-1 = -2$
\end{itemize}

Let's say $m=2k$ is an even number. Now to show that $f(v) = v_0v_1 + v_2v_3 + \cdots + v_{2k-2}v_{2k-1}$ is a bent function, we notice the following:
\begin{align*}
    \tilde{F}(u) &&=&& \sum_{v \in \mathbb{F}^m} (-1)^{u.v+f(v)} \\
    &&=&& \sum_v (-1)^{(u_0v_0 + \cdots u_mv_m) + (v_0v_1 + v_2v_3 + \cdots + v_{2k-2}v_{2k-1})} \\
    &&=&& \sum_v (-1)^{u_0v_0 + u_1v_1 + v_0v_1}(-1)^{u_2v_2 + u_3v_3 + v_2v_3} \ldots (-1)^{u_{2k-2}v_{2k-2} + u_{2k-1}v_{2k-1} + v_{2k-2}v_{2k-1}} \\
    &&=&& \left(\sum_{(v_0, v_1) \in \mathbb{F}^2} (-1)^{u_0v_0 + u_1v_1 + v_0v_1}\right) \left(\sum_{(v_2, v_3) \in \mathbb{F}^2}(-1)^{u_2v_2 + u_3v_3 + v_2v_3}\right) \ldots \\
    &&&& \ldots \left(\sum_{(v_{2k-2}, v_{2k-1}) \in \mathbb{F}^2}(-1)^{u_{2k-2}v_{2k-2} + u_{2k-1}v_{2k-1} + v_{2k-2}v_{2k-1}}\right) \\
\end{align*}

But as shown above, $$\sum_{(v_i, v_j) \in \mathbb{F}^2} (-1)^{u_iv_i + u_jv_j + v_iv_j} = 1 + (-1)^{u_i} + (-1)^{u_j} + (-1)^{u_i+u_j+1} = \pm 2$$

Hence $\tilde{F}(u) = \pm 2^{m/2}$ and so $f$ is a Bent function.
\end{proof}

\vspace{2in} %Leave more space for comments!







\end{document}

