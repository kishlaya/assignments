\documentclass[12pt]{article}

\usepackage[utf8]{inputenc}
\usepackage{latexsym,amsfonts,amssymb,amsthm,amsmath}

\setlength{\parindent}{0in}
\setlength{\oddsidemargin}{0in}
\setlength{\textwidth}{6.5in}
\setlength{\textheight}{8.8in}
\setlength{\topmargin}{0in}
\setlength{\headheight}{18pt}



\title{Coding Theory - Homework 4}
\author{Kishlaya Jaiswal}

\begin{document}

\maketitle

\vspace{0.5in}


\subsection*{Exercise 1}
\begin{proof}
First we show that 
$$max_3(r,2) = 1 + max_2(r-1,2)$$
\begin{align*}
    n &\leq max_3(r,2) \\
    &\Longleftrightarrow \exists (n,3) \text{ set in } \mathbb{F}_2^r \\
    &\Longleftrightarrow \exists [n, n-r, \geq 4] code \\
    &\Longleftrightarrow \exists [n-1, n-r, \geq 3] code \\
    &\Longleftrightarrow \exists (n-1,2) \text{ set in } \mathbb{F}_2^{r-1} \\
    &\Longleftrightarrow n \leq 1 + max_2(r-1,2)
\end{align*}
But since we know that $max_2(r-1,2) = 2^{r-1} - 1$ (no. of lines in $\mathbb{F}_2^{r-1}$), we get that $max_3(r,2) = 2^{r-1}$
\end{proof}

\subsection*{Exercise 2}
\begin{proof}
Let $V = \mathbb{F}_2^{r+1}$ and

Since every line passing through origin has exactly one more point on it (in $V$), we get that $V \setminus \{0\} \cong PG(r,2)$.

We can identify any $S \subseteq PG(r,2)$ as $S \subseteq V$.

Now, if $S$ is a maximal $n-cap$ then we know that any $3$ vectors in $S$ are linearly independent and hence by the previous problem we know that $|S| = max_3(r+1,2) = 2^r$.

Let $T = V \setminus S$. Clearly $|T| = 2^r$ as $|\mathbb{F}_2^{r+1}| = 2^{r+1}$

We shall show that $T$ is a linear subspace of $V$.
\begin{itemize}
    \item $0 \in T$. This is clear as the pullback of $S$ doesn't contain $0$ by definition.
    
    \item $v \in T \implies \lambda v \in T \forall \lambda \in \mathbb{F}_2$. If $\lambda = 0$ then we are done otherwise if $\lambda v \in S$ then since $v \sim \lambda v \implies v \in S$ which is a contradiction.
    
    \item $v, w \in T \implies v+w \in T$. Fix a $s \in S$ and consider the map 
    \begin{align*}
        f: &V \longrightarrow V \\ 
        & x \longmapsto s+x
    \end{align*}
    
    First note that $f$ is injective and hence it's a bijection.
    Now for any $s' \in S$, $f(s') = s+s' \in T$ because any 3 points in $S$ are non-collinear (and $s, s', s+s'$ are collinear points). So $f(S) \subseteq T$. But since $|S| = |T| = 2^r$, therefore $f(S) = T$ and so any point in $T$ is of the form $s+x$ for some $x \in S$.
    
    So let $v = s + s_1$ and $w = s + s_2$, where $s_1, s_2 \in S$. 
    
    Furthermore, since $s$ in the definition of $f$ was arbitrary, we get that $S+S \subseteq T$.
    
    Hence, $v+w = (s+s_1) + (s+s_2) = s_1 + s_2 \in T$ as required.
\end{itemize}

It follows that, $T'$ (projection of $T$ in $PG(r,2)$ and so $|T'| = 2^r-1$ as $T'$ doesn't contain $0$) is a $r-1$ dimensional subspace of $PG(r,2)$. Clearly, $S$ and $T$ are disjoint in their images in $PG(r,2)$ and furthermore, $|S| + |T'| = 2^r + (2^r - 1) = 2^{r+1} - 1 = |PG(r,2)|$. And therefore, $S$ is obtained from the complement of some hyperplane in $PG(r,2)$.
\end{proof}

\subsection*{Exercise 3}
\begin{proof}
Let $\mathbb{F}_q^* = \{a_1, a_2, \ldots a_{q-1}\}$. Consider
$$H = \begin{pmatrix}
1 & 1 & 1 & \ldots & 1 & 1 & 0 & 0 \\
a_1 & a_2 & a_3 & \ldots & a_{q-1} & 0 & 1 & 0 \\
a_1^2 & a_2^2 & a_3^2 & \ldots & a_{q-1}^2 & 0 & 0 & 1 \\
\end{pmatrix}$$

We will show that $H$ when considered as a parity check matrix, gives a $[q+2, q-1, 4]$ code iff $q$ is even prime power. Otherwise if $q$ is odd prime power, then we get $[q+2, q-1, \geq 5]$ code.

First, it is clear that $n=q+2$ and that the rank of $H$ is $3$ and so $k = q-1$. 

Now, if we consider the following submatrix $H'$ of $H$:
$$H' = \begin{pmatrix}
1 & 1 & 0 \\
a_i & a_j & 1 \\
a_i^2 & a_j^2 & 0 \\
\end{pmatrix}$$
Then $|H'| = (a_j^2-a_i^2)$.

If $q$ is even prime power, then $|H'| = (a_j^2-a_i^2) = (a_j-a_i)^2 \neq 0$ as $a_i \neq a_j$.

But if $q$ is odd prime power, then $|H'| = (a_j^2-a_i^2) = (a_j-a_i)(a_j+a_i) = 0$, whenever $a_i$ and $a_j$ are additive inverses of each other, which are distinct.

(Note that, for characteristic $2$ fields, additive inverse of $x$ is $x$ itself).

\end{proof}

\vspace{2in} %Leave more space for comments!



\end{document}

