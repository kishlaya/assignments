\documentclass[12pt]{article}

\usepackage[utf8]{inputenc}
\usepackage{latexsym,amsfonts,amssymb,amsthm,amsmath,mathrsfs}
\usepackage[makeroom]{cancel}
\usepackage {tikz}
\usetikzlibrary {positioning}

\setlength{\parindent}{0in}
\setlength{\oddsidemargin}{0in}
\setlength{\textwidth}{6.5in}
\setlength{\textheight}{8.8in}
\setlength{\topmargin}{0in}
\setlength{\headheight}{18pt}

\makeatletter
\newcommand\mathcircled[1]{%
  \mathpalette\@mathcircled{#1}%
}
\newcommand\@mathcircled[2]{%
  \tikz[baseline=(math.base)] \node[draw,circle,inner sep=1pt] (math) {$\m@th#1#2$};%
}
\makeatother

\title{Topics in Algorithms - Assignment 1}
\author{Kishlaya Jaiswal}

\begin{document}

\maketitle

\vspace{0.5in}


\subsection*{Exercise 1}
\begin{proof}
Consider the following preference lists:

\begin{minipage}{0.45\textwidth}
\begin{align*}
    m_1: w_1, w_2, w_3 \\
    m_2: w_2, w_3, w_1 \\
    m_3: w_3, w_1, w_2
\end{align*}
\end{minipage}%
\hfill
\begin{minipage}{0.45\textwidth}
\begin{align*}
    w_1: m_2, m_3, m_1 \\
    w_2: m_3, m_1, m_2 \\
    w_3: m_1, m_2, m_3
\end{align*}
\end{minipage}
\newline

Then the \textsl{man-optimal} matching is:

\begin{minipage}{0.45\textwidth}
\begin{align*}
    m_1: \mathcircled{w_1}, w_2, w_3 \\
    m_2: \mathcircled{w_2}, w_3, w_1 \\
    m_3: \mathcircled{w_3}, w_1, w_2
\end{align*}
\end{minipage}%
\hfill
\begin{minipage}{0.45\textwidth}
\begin{align*}
    w_1: m_2, m_3, \mathcircled{m_1} \\
    w_2: m_3, m_1, \mathcircled{m_2} \\
    w_3: m_1, m_2, \mathcircled{m_3}
\end{align*}
\end{minipage}
\newline

The \textsl{woman-optimal} matching is:

\begin{minipage}{0.45\textwidth}
\begin{align*}
    m_1: w_1, w_2, \mathcircled{w_3} \\
    m_2: w_2, w_3, \mathcircled{w_1} \\
    m_3: w_3, w_1, \mathcircled{w_2}
\end{align*}
\end{minipage}%
\hfill
\begin{minipage}{0.45\textwidth}
\begin{align*}
    w_1: \mathcircled{m_2}, m_3, m_1 \\
    w_2: \mathcircled{m_3}, m_1, m_2 \\
    w_3: \mathcircled{m_1}, m_2, m_3
\end{align*}
\end{minipage}
\newline

But we have another stable matching:

\begin{minipage}{0.45\textwidth}
\begin{align*}
    m_1: w_1, \mathcircled{w_2}, w_3 \\
    m_2: w_2, \mathcircled{w_3}, w_1 \\
    m_3: w_3, \mathcircled{w_1}, w_2
\end{align*}
\end{minipage}%
\hfill
\begin{minipage}{0.45\textwidth}
\begin{align*}
    w_1: m_2, \mathcircled{m_3}, m_1 \\
    w_2: m_3, \mathcircled{m_1}, m_2 \\
    w_3: m_1, \mathcircled{m_2}, m_3
\end{align*}
\end{minipage}
\end{proof}

\subsection*{Exercise 2}
\begin{proof}
Consider the following preference lists (with the circled people as \textsl{man-optimal} stable matching):

\begin{minipage}{0.45\textwidth}
\begin{align*}
    m_1: w_1, \mathcircled{w_2}, w_3 \\
    m_2: w_2, \mathcircled{w_1}, w_3 \\
    m_3: w_2, \mathcircled{w_3}, w_1
\end{align*}
\end{minipage}%
\hfill
\begin{minipage}{0.45\textwidth}
\begin{align*}
    w_1: \mathcircled{m_2}, m_1, m_3 \\
    w_2: \mathcircled{m_1}, m_3, m_2 \\
    w_3: \mathcircled{m_3}, m_1, m_2 \\
\end{align*}
\end{minipage}
\newline

But clearly, we can re-pair $(m_1, w_1)$ and $(m_2, w_2)$ to get an unstable matching which is not \textsl{pareto-optimal}:

\begin{minipage}{0.45\textwidth}
\begin{align*}
    m_1: \mathcircled{w_1}, w_2, w_3 \\
    m_2: \mathcircled{w_2}, w_1, w_3 \\
    m_3: w_2, \mathcircled{w_3}, w_1
\end{align*}
\end{minipage}%
\hfill
\begin{minipage}{0.45\textwidth}
\begin{align*}
    w_1: m_2, \mathcircled{m_1}, m_3 \\
    w_2: m_1, m_3, \mathcircled{m_2} \\
    w_3: \mathcircled{m_3}, m_1, m_2 \\
\end{align*}
\end{minipage}
\newline
\end{proof}

\subsection*{Exercise 3}
\begin{proof}
Consider the following preference lists (with the circled people as \textsl{man-optimal} stable matching), then no man gets matched to their first choice.

\begin{minipage}{0.45\textwidth}
\begin{align*}
    m_1: w_1, \mathcircled{w_2}, w_3 \\
    m_2: w_2, \mathcircled{w_1}, w_3 \\
    m_3: w_2, \mathcircled{w_3}, w_1
\end{align*}
\end{minipage}%
\hfill
\begin{minipage}{0.45\textwidth}
\begin{align*}
    w_1: \mathcircled{m_2}, m_1, m_3 \\
    w_2: \mathcircled{m_1}, m_3, m_2 \\
    w_3: \mathcircled{m_3}, m_1, m_2 \\
\end{align*}
\end{minipage}
\newline

The following preference list gives a counter-example for: there doesn't always exist a stable matching that matches some person to their first choice (because both the \textsl{man-optimal} and \textsl{woman-optimal} matchings are the same) as follows:

\begin{minipage}{0.45\textwidth}
\begin{align*}
    m_1: w_1, \mathcircled{w_2}, w_3, w_4 \\
    m_2: w_2, w_1, \mathcircled{w_3}, w_4 \\
    m_3: w_2, \mathcircled{w_1}, w_3, w_4 \\
    m_4: w_3, \mathcircled{w_4}, w_1, w_2 \\
\end{align*}
\end{minipage}%
\hfill
\begin{minipage}{0.45\textwidth}
\begin{align*}
    w_1: m_4, \mathcircled{m_3}, m_2, m_1 \\
    w_2: m_4, \mathcircled{m_1}, m_3, m_2 \\
    w_3: m_1, \mathcircled{m_2}, m_3, m_4 \\
    w_4: m_1, \mathcircled{m_4}, m_2, m_3 \\
\end{align*}
\end{minipage}
\newline
\end{proof}


\subsection*{Exercise 4}
\begin{proof}
\textbf{(a)} Suppose a pair $(m,w)$ is deleted during a run of the GS algorithm (with men-proposing), that is $m$ was deleted from $w's$ list and $w$ was deleted from $m's$ list. This means some man $m'$ must have proposed $w$ who was more preferable (to $w$) than $m$.

But we know that during a run of GS algo, if a woman gets engaged to someone during the run, then she can only get a better partner later during the run. Therefore, $w$ is engaged to $m'$ (or better) in the \textsl{woman-pessimal} matching (GS outpust \textsl{man-optimal}, \textsl{women-pessimal} matching). Thus, $m$ can't be a partner of $w$ in any stable matching. Hence, $(m,w)$ can't form a stable pair.

Furthermore, since in any matching $w$ is engaged to someone strictly better $m$, $(m,w)$ can't block any stable matching.

Hence the stable matchings are not affected by this operation.
\newline

\textbf{(b) No.} Consider the following preference lists (with the circled people as \textsl{man-optimal} stable matching):

\begin{minipage}{0.45\textwidth}
\begin{align*}
    m_1: \mathcircled{w_1}, w_2, w_3 \\
    m_2: \cancel{w_1}, \mathcircled{w_2}, w_3 \\
    m_3: \cancel{w_1}, \cancel{w_2}, \mathcircled{w_3} \\
\end{align*}
\end{minipage}%
\hfill
\begin{minipage}{0.45\textwidth}
\begin{align*}
    w_1: \mathcircled{m_1}, \cancel{m_2}, \cancel{m_3} \\
    w_2: m_1, \mathcircled{m_2}, \cancel{m_3} \\
    w_3: m_1, m_2, \mathcircled{m_3} \\
\end{align*}
\end{minipage}
\newline

We can check that \textsl{woman-optimal} matching is also the same and hence there is exactly one stable matching. This tells us that the "not-crossed-out" pair $(m_1, w_2)$ cannot appear in any of the stable matchings.
\newline

\textbf{(c) No.} Consider the following preference lists:

\begin{minipage}{0.45\textwidth}
\begin{align*}
    m_1: w_3, w_2, w_1 \\
    m_2: w_3, w_1, w_2 \\
    m_3: w_2, w_1, w_3 \\
\end{align*}
\end{minipage}%
\hfill
\begin{minipage}{0.45\textwidth}
\begin{align*}
    w_1: m_1, m_3, m_2 \\
    w_2: m_2, m_1, m_3 \\
    w_3: m_3, m_1, m_2 \\
\end{align*}
\end{minipage}
\newline

The reduced list after first run of GS (man-proposing) will be (with the circled people as \textsl{man-optimal} stable matching):

\begin{minipage}{0.45\textwidth}
\begin{align*}
    m_1: \mathcircled{w_3}, w_2, w_1 \\
    m_2: \mathcircled{w_1}, w_2 \\
    m_3: \mathcircled{w_2}, w_1, w_3 \\
\end{align*}
\end{minipage}%
\hfill
\begin{minipage}{0.45\textwidth}
\begin{align*}
    w_1: m_1, m_3, \mathcircled{m_2} \\
    w_2: m_2, m_1, \mathcircled{m_3} \\
    w_3: m_3, \mathcircled{m_1} \\
\end{align*}
\end{minipage}
\newline

Further reducing these lists using GS (woman-proposing) we get (with the circled people as \textsl{woman-optimal} stable matching):

\begin{minipage}{0.45\textwidth}
\begin{align*}
    m_1: w_3, w_2, \mathcircled{w_1} \\
    m_2: w_1, \mathcircled{w_2} \\
    m_3: w_2, w_1, \mathcircled{w_3} \\
\end{align*}
\end{minipage}%
\hfill
\begin{minipage}{0.45\textwidth}
\begin{align*}
    w_1: \mathcircled{m_1}, m_3, m_2 \\
    w_2: \mathcircled{m_2}, m_1, m_3 \\
    w_3: \mathcircled{m_3}, m_1 \\
\end{align*}
\end{minipage}
\newline

But now note that, $(m_1, w_2)$ cannot be a stable pair. For the sake of contradiction, suppose it is. Then the only option is to pair $m_2$ with $w_1$ and $m_3$ with $w_3$. But then $(m_3, w_1)$ is a blocking pair.

\end{proof}

\subsection*{Exercise 5}
\textbf{(i)} Let $R_1$, $R_2$ and $R_3$ be any three stable matchings.

\textbf{Claim: } If $(x,y) \in R_1$ but $(x,y) \not \in R_2$, then one of $\{x,y\}$ prefers $R_1$ and the other one prefers $R_2$.

\textsl{Proof: } Let $S_1$ be the set of people who prefer $R_1$ to $R_2$ and $S_2$ be the set of people who prefer $R_2$ to $R_1$. Then for any $x \in S_1$, $R(x) \in S_2$ otherwise $(x,y)$ blocks $R_2$ and so $|S_1| \leq |S_2|$. Similarly, for any $x \in S_2$, $R_2(x) \in S_1$ and so $|S_1| \leq |S_2|$. Hence $|S_1| = |S_2|$ and it follows that one of $\{x,y\}$ prefers $R_1$ and the other one prefers $R_2$.
\newline

To conclude that taking medians results in a matching, let $y$ be the median choice of $x$. Then note that if $x$ prefers $R_i$ to $R_j$ to $R_k$, then $y$ prefers $R_k$ to $R_j$ to $R_i$ and hence $x$ is the median choice of $y$.

Thus, taking the median partners w.r.t $R_1$, $R_2$ and $R_3$ results in a matching. Now we shall show that it is indeed a stable matching.

Suppose not, then there is a blocking pair $(a,b)$ w.r.t to the median matching. Let's say $R_i(a)$ was the median partner of $a$ and $R_j(b)$ was the median partner of $b$. Then
$$a: \ldots b \ldots R_i(a) \ldots $$
$$b: \ldots a \ldots R_j(b) \ldots $$
Now if $(i=j)$ or $(b: \ldots a \ldots R_j(b) \ldots R_i(b) \ldots)$ then $(a,b)$ blocks $R_i$ which is a contradiction. So, we must have
$$b: \ldots R_i(b) \ldots a \ldots R_j(b) \ldots $$
Similarly, 
$$a: \ldots R_j(a) \ldots b \ldots R_i(a) \ldots $$
Consider third matching $R_k$. Since $R_i(a)$ was median partner of $a$ and $R_j(b)$ was median partner of $b$, we must have
$$a: \ldots R_j(a) \ldots b \ldots R_i(a) \ldots R_k(a) \ldots $$
$$b: \ldots R_i(b) \ldots a \ldots R_j(b) \ldots R_k(b) \ldots $$
And hence $(a,b)$ blocks $R_k$, which is a contradiction to the stability of $R_k$.
\newline

\textbf{(ii)} Consider the set $\mathscr{S} = \{P(R_0) \oplus P(R) | \text{ stable matching } R\}$. We will first show that $\mathscr{S}$ is closed under intersection.

For that let us setup some notation first. Let $\mathscr{R} = \{R_i \mid  0 \leq i \leq m\}$ be the set of all stable matchings, and let $P_i = P(R_0) \oplus P(R_i)$ then $\mathscr{S} = \{P_i \mid 0 \leq i \leq m\}$. 

Now, let $P_i, P_j \in S$. Let $R$ be median (stable) matching of $R_0, R_i, R_j$. We claim that $P_i \cap P_j = P(R_0) \oplus P(R)$.

To show $P_i \cap P_j \subseteq P(R_0) \oplus P(R)$, let $(x,y) \in P_i \cap P_j$ then:

\textbf{Case 1:} $x$ prefers $R_0$ over $R_i$ $\implies$ $x$ prefers $R_0$ over $R_j$ as $(x,y) \in P_j$. WLOG assume $x$ prefers $R_i$ over $R_j$
$$x: \ldots R_0(x) \ldots y \ldots R_i(x) \ldots R_j(x) \ldots$$
Thus, median partner of $x$ is $R_i(x)$ and hence $(x,y) \in P(R_0) \oplus P(R)$.

\textbf{Case 2:} $x$ prefers $R_i$ over $R_0$ $\implies$ $x$ prefers $R_j$ over $R_0$ as $(x,y) \in P_j$. WLOG assume $x$ prefers $R_i$ over $R_j$
$$x: \ldots R_i(x) \ldots R_j(x) \ldots y \ldots R_0(x) \ldots$$
Thus, median partner of $x$ is $R_j(x)$ and hence $(x,y) \in P(R_0) \oplus P(R)$.

\textbf{Case 3:} $x$ is indifferent between $R_0$ and $R_i$. This cannot happen because otherwise $x$'s list would be empty in $P_i$.
\newline

For the other direction, let $(x,y) \in P(R_0) \oplus P(R)$. Since $R$ is the median of $R_i, R_j, R_0$, we have the following three cases:

\textbf{Case 1:} $R(x) = R_i(x)$ then we have either 
$$x: \ldots R_j(x) \ldots R_i(x) \ldots y \ldots R_0(x) \ldots$$
OR
$$x: \ldots R_0(x) \ldots y \ldots R_i(x) \ldots R_j(x) \ldots$$
But in both the cases, it is easy to see that $(x,y) \in P_i \cap P_j$

\textbf{Case 2:} $R(x) = R_j(x)$ Similar to the above case.

\textbf{Case 3:} $R(x) = R_0(x)$ This cannot happen as $x$'s list will be empty in $P(R_0) \oplus P(R)$.
\newline

Therefore, $(\mathscr{S}, \subseteq)$ is a poset (under subset inclusion) closed under intersection and hence forms a meet-semi-lattice, in which $\text{empty set } = P(R_0) \oplus P(R_0)$ is the minimal element.

Consider the map $\varphi: \mathscr{R} \rightarrow \mathscr{S}$ defined as $R_i \mapsto P_i$. Then $\varphi$ is clearly a bijection as $P_i \neq P_j \forall i \neq j$. Hence, $\varphi^{-1}$ induces a meet-semi-lattice structure on set of all stable matchings $\mathscr{R}$.

\subsection*{Exercise 6}
\begin{proof}.

\textbf{(a)} Notice that regret of $T$ is defined as the maximum regret edge, say $(x,y)$ where $y = l_T(x)$. If in phase 2, the rotation that $y$ leads to is never eliminated, then the pair $\{x,y\}$ will never be deleted and hence $(x,y)$ will be partners in the stable matching generated at the end of phase 2 algorithm. Thus, the regret of $T$ remains unaltered and hence we obtain the maximum regret stable matching among all the matchings embedded in $T$.
\newline

\textbf{(b)} Instead of choosing an arbitrary rotation, we first find the maximum regret edge of $T$, say $(x,y)$ where $y=l_T(x)$. If $y$ is the only entry in $x$'s list, then in any stable matching (embedded in $T$), $(x,y)$ will be partners and so in particular any minimum regret stable matching has this edge; otherwise, we find the rotation that $y$ leads to, that is, consider $y$'s list: $x, x', \ldots$, then we can find $y'$ such that $y': x', x'', \ldots$. Repeating this process we can find a rotation. We eliminate \textsl{this particular rotation} in the next iteration. Either $(x,y)$ was a part of this rotation and hence gets deleted or we can repeat this process until $(x,y)$ gets deleted in the subsequent iterations. Having removed the maximum regret edge, we choose the next maximum regret edge and repeat the above process.

The only amendment we made was in phase 2 of algorithm where we find a maximum regret edge before finding/deleting an arbitrary rotation. This new sub-routine to find a maximum regret edge takes time $O(n^2)$ and it doesn't interfere with find/delete of rotation which takes time $O(n^2)$ already. Hence the total time taken by the algorithm is $O(n^2) + O(n^2) = O(n^2)$.
\newline

\textbf{(c)} We argue by contradiction. Let $M_0$ be any minimum regret matching and $M$ be the output of our modified algorithm such that regret of $M$ is strictly more than $M_0$ ($regret(M_0) < regret(M)$). Consider the sequence of table reductions we followed to get $M$: $T_0 \rightarrow T_1 \rightarrow T_2 \rightarrow \ldots \rightarrow M$. Let $T_i$ be the last table in this sequence which contains $M_0$. We can find such a table because we started with the phase 1 table $T_0$ which contains both $M_0$ and $M$. Therefore, $T_{i+1}$ doesn't contain $M_0$. This means, when going from $T_i$ to $T_{i+1}$, we must have deleted a rotation which contains an edge of $M_0$.
Let's say, when at $T_i$, we picked a maximum regret edge $(x,y)$ w.r.t to $T_i$ and the rotation $\rho$ leading to it, which we deleted to obtain $T_{i+1}$. By the above observation, $\rho$ contains an $M_0$ edge say $(x',y')$. But then $regret(M_0) \geq regret((x,y)) \geq regret(x',y') \geq regret(M)$, which leads us to a contradiction.
\end{proof}

\vspace{2in} %Leave more space for comments!

\end{document}

