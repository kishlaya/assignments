\documentclass[12pt]{article}

\usepackage[utf8]{inputenc}
\usepackage{latexsym,amsfonts,amssymb,amsthm,amsmath}
\usepackage[makeroom]{cancel}
\usepackage {tikz}
\usetikzlibrary {positioning}

\setlength{\parindent}{0in}
\setlength{\oddsidemargin}{0in}
\setlength{\textwidth}{6.5in}
\setlength{\textheight}{8.8in}
\setlength{\topmargin}{0in}
\setlength{\headheight}{18pt}


\title{Complexity Theory - Assignment 1}
\author{Kishlaya Jaiswal, Satya Prakash Nayak}

\begin{document}

\maketitle

\vspace{0.5in}


\subsection*{Exercise 1}
\begin{proof}
We shall prove the contrapositive that $P = NP \implies EXP = NEXP$. So assume $P=NP$ then it suffices to show $NEXP \subseteq EXP$.

Let $L \in NEXP$, then there exist a non-deterministic turing machine $M$ with running time $T(x) = O(2^{|x|^c})$ that decides $L$.
Define $L_p = \{x10^{T(x)} \mid x \in L\}$.\\

We claim that $L_p \in NP$. For any $y \in L_p$, we have $|y| = |x| + T(x) + 1$. Define NTM $M'$ that does the following: on input $y$ it throws out the last $10^{T(x)}$ parts and run $M$ on $x$ (remaining part) and does whatever $M$ does. Since $M$ runs in $T(x)$ time, $M'$ is a NP machine. Therefore, $L_p \in NP \implies L_p \in P$, so there exists a deterministic poly-time machine $D$ which decides $L_p$. Now we can construct a deterministic machine $D'$ which on input $x$, will pad it with $10^{T(x)}$ and check whether $x10^{T(x)} \in L_p$ using the machine $D$ on input $x$. Hence, $L \in EXP$.
\end{proof}

\subsection*{Exercise 2}
\begin{proof}
We shall show $SAT \leq_p HALT$.

Let $N$ be the poly-time $NTM$ that decides $SAT$. Consider the modified $NTM$ $M$ which on input $x$, accepts if $N$ accepts $x$ and otherwise if $N$ rejects then $M$ goes into a loop, and hence never halts. Let $\alpha$ be the encoding for this machine $M$.

Given a SAT-instance $\phi$ encoded as $x$, consider the string $(\alpha, x)$ as an instance of $HALT$ problem. (Note that: $M_\alpha = M$)

Suppose $\phi \in SAT$, then there exists a satisfying assignment and so $N$ accepts $x$ $\implies$ $M = M_\alpha$ halts and accepts $x$ and therefore $(\alpha, x) \in HALT$. Conversely, $\phi \not \in SAT$, then $N$ rejects $x$ $\implies$ $M = M_\alpha$ doesn't halt on $x$ and therefore $(\alpha, x) \not \in HALT$.

Thus, $HALT$ is $NP$-hard.
\newline

Furthermore, $HALT \not \in NP$ because otherwise, if there exists a $NTM$ which decides $HALT$, then given a machine $M_\alpha$ and an input $x$, we could decide if $M_\alpha$ halts on input $x$, hence solving the HALTING problem. So $HALT$ is not $NP$-complete.
\end{proof}

\subsection*{Exercise 3}
\begin{proof}
$L_1 \in NP \implies \exists$ a polynomial $p_1$ and a poly-time predicate $B_1$ such that $x \in L_1 \iff \exists w, |w| \leq p_1(|x|) \wedge B_1(w,x) = 1$

$L_2 \in NP \implies \exists$ a polynomial $p_2$ and a poly-time predicate $B_2$ such that $x \in L_2 \iff \exists w, |w| \leq p_2(|x|) \wedge B_2(w,x) = 1$
\newline

To show $L_1 \cup L_2 \in NP$, consider the polynomial $p = p_1 + p_2$ and the poly-time predicate $B = B_1 \vee B_2$, then: $x \in L_1 \cup L_2 \iff (x \in L_1) \vee (x \in L_2) \iff (\exists w, |w| \leq p_1(|x|) \wedge B_1(w,x) = 1) \vee (\exists w, |w| \leq p_2(|x|) \wedge B_2(w,x) = 1) \iff \exists w, |w| \leq p(|x|) \wedge B(w,x) = 1$
\newline

To show $L_1 \cap L_2 \in NP$, consider the polynomial $p = p_1 + p_2$ and the poly-time predicate $B(w_1 \# w_2) = B_1(w_1) \wedge B_2(w_2)$, then: $x \in L_1 \cap L_2 \iff (x \in L_1) \wedge (x \in L_2) \iff (\exists w_1, |w_1| \leq p_1(|x|) \wedge B_1(w_1,x) = 1) \wedge (\exists w_2, |w_2| \leq p_2(|x|) \wedge B_2(w_2,x) = 1) \iff \exists w, w = w_1 \# w_2, |w| \leq p(|x|) \wedge B(w,x) = 1$
\end{proof}


\subsection*{Exercise 4}
\begin{proof}.
It is clear that $TAUT \in coNP$ because given any NO instance $\phi(x)$ of $TAUT$, a short certificate for verification is an assignment of variables $x$ such that $\phi(x) = 0$. Clearly such an assignment can be expressed in $O(n\lg n)$ size (where $n$ = number of variables).

To show that $TAUT$ is $coNP-complete$, we note that, for any two $NP$ languages $A$ and $B$, $A \leq_p B \iff \overline{A} \leq_p \overline{B}$.

Thus, if $L$ is $NP-complete$, then $\overline{L}$ is $coNP-complete$ because firstly, $\overline{L} \in coNP$ by definition and for any $L' \in coNP$, $\overline{L'} \leq_p L \implies L' \leq_p \overline{L}$. So, we conclude that $\overline{3SAT}$ is $coNP-complete$.

To show that $TAUT$ is $coNP-complete$, we show that $\overline{3SAT} \leq_p TAUT$. Given any formula $\varphi$, consider the formula $\overline{\varphi}$. This formula can be constructed in $O(|\varphi|)$ time. Now, $\varphi$ doesn't have a satisfying assignment iff $\overline{\varphi}$ is a tautology.
\newline

Suppose $NP=coNP$. Since $3SAT$ is $NP-complete$, for every $NP$ language $L$, $L \leq_p 3SAT$. But $TAUT \in coNP = NP \implies TAUT \leq_p 3SAT$. Similarly, since $TAUT$ is $coNP-complete$, for every $coNP$ language $L$, $L \leq_p TAUT$. But $3SAT \in NP = coNP \implies 3SAT \leq_p TAUT$.

Conversely, let $L \in NP$, then $L \leq_p 3SAT \leq_p TAUT$. So given a NO instance of the problem $L$, we reduce it to a NO instance of $TAUT$ problem and hence we have a short certificate for the NO instances of $L$ as well implying that $L \in coNP$. Similarly, let $L \in coNP$, then $L \leq_p TAUT \leq_p 3SAT$. So given a YES instance of the problem $L$, we reduce it to a YES instance of $3SAT$ problem and hence we have a short certificate for the YES instances of $L$ as well implying that $L \in NP$. Thus, $NP=coNP$.
\end{proof}

\subsection*{Exercise 5}
\begin{proof}
Suppose unary $NP \subseteq P$.

Let $L \in NEXP$, then there exist a non-deterministic turing machine $M$ with running time $T(x) = 2^{|x|^c}$ that decides $L$. 

Define $L_u = \{Unary(x) \mid x \in L\}$.\\

Observe that if $x \in L$, then $x$ is a binary string of length $|x|$ and so the unary representation of $x$ will have size atmost $2^{|x|}$.

We claim that $L_u \in NP$. For that we construct a machine $M'$ which on input $1^y$, converts $y$ into it's binary $x$ and then simulate $M$ on $x$. $M'$ accepts $1^y$ iff $M$ accepts $x$. Since $|1^y| = y < 2^{|x|}$, $M$ requires $2^{|x|} + 2^{|x|^c} = O(2^{|x|^c}) = O(y^c)$ time and hence $L(M') = L_u \in NP$.

Therefore, $L_u \in P$, so there exists a deterministic poly-time machine $D$ which decides $L_u$. Now we can construct a deterministic machine $D'$ which on input $x$, will construct $Unary(x)$ and check if it is in $L_u$ using the machine $D$. Hence, $L \in EXP$.
\end{proof}

\subsection*{Exercise 6}
\begin{proof}
Since $P=NP$, we claim that $P = coNP$: $L \in coNP \implies \overline{L} \in NP \implies \overline{L} \in P \implies L \in coP = P$.

Thus $TAUT \in P$ and so there exists a poly-time algorithm $A$ such that $A(\varphi(y)) = 1$ iff $\varphi(y)$ is a tautology.

We shall show that $\Sigma_2 SAT \in NP$. Consider a YES instance of $\Sigma_2 SAT$, that is a formula $\psi = \exists x \forall y (\phi(x,y) = 1)$ which admits a $x_0$ such that $\phi(x_0,y)$ is a tautology.

Then a short certificate for $\psi \in \Sigma_2 SAT$ is $x_0$; because clearly $|x_0| \leq |\psi|$ and in poly-time we can check if $\phi(x_0,y)$ is a tautology by calling $A(\phi(x_0, y))$.

In other words, $\psi \in \Sigma_2 SAT \iff \exists x_0$ such that $|x_0| \leq |\psi|$ and $A(\phi(x_0,y))=1$.

Hence $\Sigma_2 SAT \in P$.
\end{proof}

\vspace{2in} %Leave more space for comments!



\end{document}
