 \documentclass{article}
\usepackage[margin=1in]{geometry}
\setlength{\parindent}{0in}

\usepackage[utf8]{inputenc}

\usepackage{latexsym,amsfonts,amssymb,amsthm,amsmath}
\usepackage {tikz}
\usetikzlibrary {positioning}
\usetikzlibrary{quantikz}

\usepackage{braket}

\newcommand{\norm}[1]{\left\lVert#1\right\rVert}

\title{Quantum Computing - Assignment 1}
\author{Kishlaya Jaiswal}

\begin{document}

\maketitle


\subsection*{Exercise 1}
\begin{proof}
Observe that
$$\norm{ \ket{\psi} }^2 = \braket{\psi|\psi} = \big(\ket{\psi}\big)^{\dag} \big(\ket{\psi}\big)$$
$U$ is unitary that is $U^{\dag} U = I$, and hence
$$\norm{ U\ket{\psi} }^2 = \big(U\ket{\psi}\big)^{\dag} \big(U\ket{\psi}\big) = \bra{\psi} U^\dag U \ket{\psi} = \bra{\psi} I \ket{\psi} = \braket{\psi|\psi} = \norm{ \ket{\psi} }^2$$

Since $\norm{.} \geq 0 \implies \norm{ U\ket{\psi} } = \norm{ \ket{\psi} }$
\end{proof}


\subsection*{Exercise 2}
\begin{proof}
\begin{align*}
    [X,Z] \ket{0} &= (XZ-ZX) \ket{0} = X\ket{0} - Z\ket{1} = \ket{1} + \ket{1} = 2 \ket{1} \\
    [X,Z] \ket{1} &= (XZ-ZX) \ket{1} = -X\ket{1} - Z\ket{0} = -\ket{0} - \ket{0} = -2 \ket{0}
\end{align*}

Hence $[X,Z] = \begin{pmatrix} 0 & -2 \\ 2 & 0\end{pmatrix}$
\end{proof}


\subsection*{Exercise 3}
\begin{proof}
$$X = \begin{pmatrix}0 & 1 \\ 1 & 0\end{pmatrix} \implies X^\dag = \begin{pmatrix}0 & 1 \\ 1 & 0\end{pmatrix} = X \text{ and } X^\dag X = \begin{pmatrix}0 & 1 \\ 1 & 0\end{pmatrix}\begin{pmatrix}0 & 1 \\ 1 & 0\end{pmatrix} = \begin{pmatrix}1 & 0 \\ 0 & 1\end{pmatrix} = I$$
$$Y = \begin{pmatrix}0 & -i \\ i & 0\end{pmatrix} \implies Y^\dag = \begin{pmatrix}0 & -i \\ i & 0\end{pmatrix} = Y \text{ and } Y^\dag Y = \begin{pmatrix}0 & -i \\ i & 0\end{pmatrix}\begin{pmatrix}0 & -i \\ i & 0\end{pmatrix} = \begin{pmatrix}1 & 0 \\ 0 & 1\end{pmatrix} = I$$
$$Z = \begin{pmatrix}1 & 0 \\ 0 & -1\end{pmatrix} \implies Z^\dag = \begin{pmatrix}1 & 0 \\ 0 & -1\end{pmatrix} = Z \text{ and } Z^\dag Z = \begin{pmatrix}1 & 0 \\ 0 & -1\end{pmatrix}\begin{pmatrix}1 & 0 \\ 0 & -1\end{pmatrix} = \begin{pmatrix}1 & 0 \\ 0 & 1\end{pmatrix} = I$$

Thus Pauli matrices are Hermitian and Unitary. And,

% Suppose $A$ is any Hermitian matrix, then for any non-zero eigenvector $v$ with eigenvalue $\lambda$: 
% $$\bar\lambda \braket{v, v} = \braket{\lambda v, v} = \braket{Av, v} = \braket{v, A^\dag v} = \braket{v, Av} = \braket{v, \lambda v} = \lambda \braket{v, v}$$
% So, eigenvalues of Hermitian matrices are real. 

% Suppose $A$ is any Unitary matrix, then for any non-zero eigenvector $v$ with eigenvalue $\lambda$: 
% $$\bar \lambda \braket{v, Av} = \braket{\lambda v, Av} = \braket{Av, Av} = \braket{v, A^\dag A v} = \braket{v, v}$$

% But $\bar \lambda \braket{v, Av} = \lambda\bar \lambda \braket{v, v}$ and hence $|\lambda|^2 = 1$. So eigenvalues of Unitary matrices lie on the unit circle.

% Since the only real numbers satisfying $|\lambda|^2 = 1$ are $\pm 1$, so only possible eigenvalues of Pauli matrices are $\pm 1$ and in particular

$$X\begin{pmatrix}1/\sqrt{2} \\ 1/\sqrt{2}\end{pmatrix} = \begin{pmatrix}1/\sqrt{2} \\ 1/\sqrt{2}\end{pmatrix}, X\begin{pmatrix}1/\sqrt{2} \\ -1/\sqrt{2}\end{pmatrix} = - \begin{pmatrix}1/\sqrt{2} \\ -1/\sqrt{2}\end{pmatrix}$$

$$Y\begin{pmatrix}1/\sqrt{2} \\ i/\sqrt{2}\end{pmatrix} = \begin{pmatrix}1/\sqrt{2} \\ i/\sqrt{2}\end{pmatrix}, Y\begin{pmatrix}1/\sqrt{2} \\ -i/\sqrt{2}\end{pmatrix} = -\begin{pmatrix}1/\sqrt{2} \\ -i/\sqrt{2}\end{pmatrix}$$

$$Z\begin{pmatrix}1 \\ 0\end{pmatrix} = \begin{pmatrix}1 \\ 0\end{pmatrix}, Z\begin{pmatrix}0 \\ 1\end{pmatrix} = -\begin{pmatrix}0 \\ 1\end{pmatrix}$$
\end{proof}

Thus the eigenvalues are of Pauli matrices are $\pm 1$.

\subsection*{Exercise 4}
\begin{proof}
\begin{align*}
  HXH \ket{0} &= HX \ket{+} = H \ket{+} = \ket{0}  \\
  HXH \ket{1} &= HX \ket{-} = H (-\ket{-}) = -\ket{1}
\end{align*}
And hence $HXH = Z$

\begin{align*}
  HZH \ket{0} &= HZ \ket{+} = H \ket{-} = \ket{1} \\  
  HZH \ket{1} &= HZ \ket{-} = H \ket{+} = \ket{0}
\end{align*}
And hence $HZH = X$
\end{proof}


\subsection*{Exercise 5}
\begin{proof}
From the above exercise $3$, we know that $\ket{+}$ is an eigenvector of $X$ with eigenvalue $1$ and $\ket{-}$ is an eigenvector of $X$ with eigenvalue $-1$, that is $X = \ket{+}\bra{+} - \ket{-}\bra{-}$

Hence $\{\ket+ \bra+, \ket- \bra-\}$ is an eigenbasis for $X$ and so $\ket+ \bra+$ and $\ket- \bra-$ are the measurement operators corresponding to a measurement of $X$ observable.
\end{proof}


\subsection*{Exercise 6}
\begin{proof}
First we check that $\frac{1}{\sqrt{2}}\big(\ket{00} + \ket{11}\big) = \frac{1}{\sqrt{2}}\big(\ket{++} + \ket{--}\big)$ indeed.

\begin{align*}
    \ket{++} + \ket{--} &= \frac{1}{2}(\ket0 + \ket1)(\ket0 + \ket 1) + \frac{1}{2}(\ket0 - \ket1)(\ket0 - \ket 1) \\ 
    &= \frac12 (\ket{00} + \ket{01} + \ket{10} + \ket{11} + \ket{00} - \ket{01} - \ket{10} + \ket{11}) = \ket{00} + \ket{11}
\end{align*}

So it suffices to show that $\frac{1}{\sqrt{2}}\big( \ket{00} + \ket{11}\big)$ is an entangled state. \\

Suppose not and so it can be written as $\ket{\psi} \otimes \ket{\phi}$ where $\ket{\psi} = \alpha \ket0 + \beta \ket1$ and $\ket\phi = \gamma \ket0 + \delta \ket1$.

Then $\ket{\psi} \otimes \ket{\phi} = \alpha \gamma \ket{00} + \alpha \delta \ket{01} + \beta \gamma \ket{10} + \beta \delta \ket{11} = \frac{1}{\sqrt{2}}\big(\ket{00} + \ket{11}\big)$ \\

Since $\{\ket{00}, \ket{01}, \ket{10}, \ket{11}\}$ is a linearly independent set in $\mathbb{C}^4$, we get $\alpha \delta = \beta \gamma = 0$ and $\alpha \gamma = \beta \delta \neq 0$ whose solution doesn't exist. Hence $\frac{1}{\sqrt{2}}\big(\ket{00} + \ket{11}\big)$ is an entangled state.
\end{proof}


\subsection*{Exercise 7}
\begin{center}
\begin{quantikz}
\lstick{$\ket{0}$} & \gate{H} & \ctrl{1} & \qw \rstick[wires=2]{$\ket\phi$} \\
\lstick{$\ket{0}$} & \qw & \targ{} & \qw 
\end{quantikz}
\end{center}

\begin{proof}
We start with $\ket\psi = \ket{00}$ state.\\


Applying $H \otimes I$ to $\ket\psi$, we get $\frac{1}{\sqrt2}\big(\ket0 + \ket1\big)\ket0 = \frac{1}{\sqrt2}\big(\ket{00} + \ket{10}\big)$\\


Applying controlled-NOT gate (where first qubit is the control and second qubit is target) to this, we finally get $\ket\phi = \frac{1}{\sqrt2}\big(\ket{00} + \ket{11}\big)$ as required.
\end{proof}


\end{document}

