 \documentclass{article}
\usepackage[margin=1in]{geometry}
\setlength{\parindent}{0in}

\usepackage[utf8]{inputenc}

\usepackage{latexsym,amsfonts,amssymb,amsthm,amsmath}
\usepackage {tikz}
\usetikzlibrary {positioning}
\usetikzlibrary{quantikz}

\usepackage{braket}

\newcommand{\norm}[1]{\left\lVert#1\right\rVert}

\title{Quantum Computing - Assignment 2}
\author{Kishlaya Jaiswal}

\begin{document}

\maketitle


\subsection*{Exercise 1}
\begin{proof} .

\begin{center}
\begin{quantikz}
\lstick{$\ket{0}^{\otimes n}$} & \gate{H^{\otimes n}} & \qw & \gate[wires=2]{U_f} & \qw & \gate{H^{\otimes n}} & \qw & \meter{} & \qw \\
\lstick{$\ket{1}$} & \gate{H} & \qw & & \qw \\
\end{quantikz}
\end{center}

where $f: \mathbb{Z}_2^n \rightarrow \mathbb{Z}_2$, $f(x) = \sum a_ix_i$ \\

First we prepare the state: 
$$\ket{0}^{\otimes n} \otimes \ket{1}$$ 

Applying $n+1$ Hadamard gates, we get: 
$$\left( \frac{1}{2^{n/2}} \sum_x \ket x \right) \otimes \left(\frac{\ket0 - \ket1}{\sqrt2}\right) = \frac{1}{2^{n/2}} \frac{1}{\sqrt2} \sum_x \ket x \ket0 - \ket x \ket1$$

Applying $U_f$, we get: 
$$\frac{1}{2^{n/2}} \frac{1}{\sqrt2} \sum_x \ket x \ket{f(x)} - \ket x \ket{\overline{f(x)}} = \left( \frac{1}{2^{n/2}} \sum_x (-1)^{f(x)} \ket x \right) \otimes \left(\frac{\ket0 - \ket1}{\sqrt2}\right)$$

Let us carefully examine the first $n$ qubits at this point.

First, say $x = x_1\ldots x_n$, then 
$$(-1)^{f(x)} \ket x = (-1)^{a_1x_1 + \cdots a_nx_n} \ket{x_1 \ldots x_n} = \bigg((-1)^{a_1x_1} \ket{x_1}\bigg) \otimes \cdots \otimes \bigg((-1)^{a_nx_n} \ket{x_n}\bigg)$$ 

Thus, the first $n$ qubits are: 
\begin{align*}
    \frac{1}{2^{n/2}} \sum_x (-1)^{f(x)} \ket x &= \bigg(\sum_{x_1} \frac{(-1)^{a_1x_1} \ket{x_1}}{\sqrt2}\bigg) \otimes \cdots \otimes \bigg(\sum_{x_n} \frac{(-1)^{a_nx_n} \ket{x_n}}{\sqrt2}\bigg) \\
    &= \bigg(\frac{\ket0 + (-1)^{a_1}\ket1}{\sqrt2} \bigg) \otimes \cdots \otimes \bigg(\frac{\ket0 + (-1)^{a_n}\ket1}{\sqrt2} \bigg) \\
\end{align*}

Now we recall that $$H\bigg(\frac{\ket0 + (-1)^x\ket1}{\sqrt2}\bigg) = \ket x$$

Thus applying $n$ Hadamard gates to the first $n$ qubits, we get:
$$\ket{a_1 \ldots a_n}$$
\end{proof}


\subsection*{Exercise 2}
\begin{proof}
First we prove by induction on $n$ that for $x \in \{0,1\}^n$, $$H^{\otimes n}\ket x = \frac{1}{2^{n/2}} \sum_{z \in \{0,1\}^n} (-1)^{x.z} \ket z $$

For $n=1$, it follows immediately as 
$$H\ket x = \frac{1}{\sqrt2} \sum_z (-1)^{x.z} \ket z = \frac{\ket0 + (-1)^x \ket1}{\sqrt2}$$

Suppose it is true for some $n \geq 1$, then for any $x' \in \{0,1\}^{n+1}$, write $x' = xx_{n+1}$ where $x \in \{0,1\}^n$, then

\begin{align*}
    H^{\otimes n+1}\ket{x'} &= H^{\otimes n+1}(\ket{x} \otimes \ket{x_{n+1}}) \\
    &= H^{\otimes n}\ket{x} \otimes H\ket{x_{n+1}} \\ 
    &= \frac{1}{2^{n/2}} \sum_{z \in \{0,1\}^n} (-1)^{x.z} \ket z \otimes \left(\frac{\ket0 + (-1)^{x_{n+1}}\ket1}{\sqrt2}\right) \\
    &= \frac{1}{2^{n+1/2}} \sum_{z \in \{0,1\}^n} (-1)^{x.z + + x_{n+1}.0} \ket z \otimes \ket0 + (-1)^{x.z + x_{n+1}.1} \ket z \otimes \ket1 \\
    &= \frac{1}{2^{n+1/2}} \sum_{z' \in \{0,1\}^{n+1}} (-1)^{x'.z'} \ket{z'}
\end{align*}

Now, we consider
\begin{align*}
    H\left(\frac{\ket x + \ket y}{\sqrt2}\right) &= H\left(\frac{\ket x + \ket{s \oplus x}}{\sqrt2}\right) \\
    &= \frac{1}{\sqrt2} H\ket{x} + \frac{1}{\sqrt2} H\ket{s \oplus x} \\
    &= \frac{1}{\sqrt2} \frac{1}{2^{n/2}} \sum_z (-1)^{x.z} \ket z +  \frac{1}{\sqrt2} \frac{1}{2^{n/2}} \sum_z (-1)^{x.z + s.z} \ket z \\
    &= \frac{1}{2^{n+1/2}} \sum_z (-1)^{x.z} \big(1 + (-1)^{s.z}\big) \ket z \\
    &= \frac{1}{2^{n-1/2}} \sum_{s.z = 0} (-1)^{x.z} \ket z \\
    &= \frac{1}{2^{n-1/2}} \sum_{z \perp s} (-1)^{x.z} \ket z
\end{align*}

\end{proof}


\subsection*{Exercise 3}
\begin{proof}
$\ket S = \sum_{s \in S} \frac{1}{2^{m/2}} \ket s$

\begin{align*}
    H\ket S &=  \sum_{s \in S} \frac{1}{2^{m/2}} H\ket s \\
    &=  \sum_{s \in S} \frac{1}{2^{m/2}} \frac{1}{2^{n/2}} \sum_{w} (-1)^{s.w} \ket w \\
    &=  \sum_{w} \frac{1}{2^{(n+m)/2}} \left(\sum_{s \in S} (-1)^{s.w} \right) \ket w \\
\end{align*}

\textbf{Claim}: $w \in S^{\perp} \implies \sum_{s \in S} (-1)^{s.w} = 2^m$

Because if $w \in S^{\perp} \implies s.w = 0, \forall s \in S \implies \sum_{s \in S} (-1)^{s.w} = |S| = 2^m$ \\

\textbf{Claim}: $w \not \in S^{\perp} \implies \sum_{s \in S} (-1)^{s.w} = 0$

Fix a basis $\{s_1, s_2, \ldots, s_m\}$ for $S$. Since $w \not \in S^{\perp}$, there exists $i$ such that $s_i.w = 1$. Now for any $s \in S$, $s = c_1s_1 + \cdots + c_ms_m$, where $c = (c_1, c_2, \ldots) \in \mathbb{Z}_2^m$. Thus,

\begin{align*}
    \sum_{s \in S} (-1)^{s.w} &= \sum_{c \in \mathbb{Z}_2^m} (-1)^{c_1 s_1.w}(-1)^{c_2 s_2.w} \ldots (-1)^{c_m s_m.w} \\
    &= (1 + (-1)^{s_1.w})(1 + (-1)^{s_2.w}) \ldots (1 + (-1)^{s_i.w}) \ldots (1 + (-1)^{s_m.w}) \\
    &= (1 + (-1)^{s_1.w})(1 + (-1)^{s_2.w}) \ldots (1 + (-1)) \ldots (1 + (-1)^{s_m.w}) \\
    &= 0
\end{align*}

Therefore, 
$$H\ket S = \frac{1}{2^{(n-m)/2}} \sum_{w \in S^{\perp}} \ket w$$

Furthermore, for any $y \in \mathbb{Z}_2^n$

\begin{align*}
    H\ket{y+S} &=  \sum_{s \in S} \frac{1}{2^{m/2}} H\ket{y+s} \\
    &=  \sum_{s \in S} \frac{1}{2^{(n+m)/2}} \sum_{w} (-1)^{(y+s).w} \ket w \\
    &=  \frac{1}{2^{(n+m)/2}} \sum_{w} (-1)^{y.w} \left(\sum_{s \in S} (-1)^{s.w} \right) \ket w \\
    &= \frac{1}{2^{(n-m)/2}} \sum_{w \in S^{\perp}} (-1)^{y.w}\ket w
\end{align*}

\end{proof}

\subsection*{Exercise 4}

\begin{proof}

Suppose $X_j = i$ is known. So $V_i = \langle w_1, \ldots w_i \rangle$ has dimension $j$.

\begin{align*}
    P[X_{j+1} = i+1] = P[w_{i+1} \not \in V_i] = 1 - P[w_{i+1} \in V_i] = 1 - \frac{2^j}{2^m}
\end{align*}

\begin{align*}
    P[X_{j+1} = i+2] &= P[w_{i+2} \not \in V_i, w_{i+1} \in V_i] \\
    &= P[w_{i+2} \not \in V_i] P[w_{i+1} \in V_i] \\
    &= \frac{2^j}{2^m} \left(1 - \frac{2^j}{2^m}\right)
\end{align*}

Similarly, 

\begin{align*}
    P[X_{j+1} = i+k] &= P[w_{i+k} \not \in V_i, w_{i+k-1} \in V_i, \ldots w_{i+1} \in V_i] \\
    &= P[w_{i+k} \not \in V_i] P[w_{i+k-1} \in V_i] \ldots P[w_{i+1} \in V_i] \\
    &= \left(1 - \frac{2^j}{2^m}\right)\left(\frac{2^j}{2^m}\right)^{k-1}
\end{align*}

Let $p_j = 2^{j}/2^m$, then

\begin{align*}
    E[X_{j+1} \mid X_{j} = i] &= \sum_{k \geq 1} (i+k) P[X_{j+1} = i+k] \\
    &= \sum_{k \geq 1} (i+k) \left(1 - p_j\right) p_j^{k-1} \\
    &= i \left(1 - p_j\right) \sum_{k \geq 1} p_j^{k-1} + (1 - p_j) \sum_{k \geq 1} k p_j^{k-1} \\
    &= i + \frac{1}{1-p_j} \\
\end{align*}

As $E[X_{j+1} \mid X_j] = X_j + \frac{1}{1-p_j}$ and $E[X_{j+1}] = E[E[X_{j+1} \mid X_j]]$, we get $E[X_{j+1}] = E[X_j] + \frac{1}{1-p_j}$

Similarly, we can calculate $E[X_1] = \frac{1}{1-p_0}$. Thus,

\begin{align*}
    E[X_m] &= E[X_{m-1}] + \frac{1}{1-p_{m-1}} \\
    &= \sum_{j=0}^{m-1} \frac{1}{1-p_j} \\
    &= \sum_{j=0}^{m-1} \frac{2^m}{2^m - 2^j} \\
    &= \sum_{j=0}^{m-1} 1 + \frac{2^j}{2^m - 2^j} \\
    &= m + \left(\sum_{j=1}^{m} \frac{1}{2^j - 1 }\right) \\
    &< m + \left(\sum_{j=1}^{m} \frac{1}{2^{j-1} }\right) & (\text{as } 1 \leq 2^{j-1} \forall j \geq 1) \\ 
    &< m + \left(\sum_{j=0}^{\infty} \frac{1}{2^{j} }\right) \\
    &= m + 2
\end{align*}

Thus, $E[X_m] < m+2$
\end{proof}

\end{document}

